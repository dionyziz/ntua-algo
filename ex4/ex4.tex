Αυτή τη φορά ο αλγόριθμος θα πρέπει να υπολογίζει και το μονοπάτι από την $s$ στην $t$ με τη μικρότερη αυτονομία. Δηλαδή αν για κάθε μονοπάτι χαρακτηρίσουμε ως βάρος του το βάρος της μέγιστης ακμής που περιέχει, αναζητούμε εκείνο το μονοπάτι που έχει, ανάμεσα σε όλα τα μονοπάτια, το ελάχιστο βάρος. Αυτό δεν είναι εφικτό με \textlatin{BFS}, καθώς ενδέχεται διαδρομές που αποτελούνται από μεγαλύτερο πλήθος ακμών να έχουν μικρότερες απαιτήσεις σε αυτονομία. Γι' αυτό θα χρησιμοποιηθεί ο αλγόριθμος του \textlatin{Prim} ώστε να υπολογιστεί το μονοπάτι με το ελάχιστο βάρος.

Αφού τρέξουμε τον αλγόριθμο του \textlatin{Prim}, η απάντηση είναι να διασχίζουμε το μονοπάτι από την κορυφή $s$ στην κορυφή $t$ στο \textlatin{MST}. Το κόστος του μονοπατιού τότε μπορεί να αναφερθεί ως η μέγιστη αυτονομία που είναι απαραίτητη για να φτάσουμε από την $s$ στην $t$.
