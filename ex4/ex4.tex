\documentclass[11pt,a4paper,oneside]{report}
\usepackage[english,greek]{babel}
\usepackage[utf8x]{inputenc}
\usepackage[noend]{algpseudocode}
\usepackage{algorithm}
\usepackage{amssymb,latexsym,amsmath,ucs,amsthm,setspace,graphicx,fancyvrb,float}
\usepackage{hyperref}
\usepackage{tkz-graph}
\usepackage{subfig}
\newtheorem*{lemma}{Λήμμα}
\newcommand{\HRule}{\rule{\linewidth}{0.5mm}}
\newcommand{\defeq}{\overset{\underset{\mathrm{def}}{}}{=}}
\makeatletter
\DeclareMathOperator*{\argmin}{arg\,min}

\makeatother
\raggedbottom
\begin{document}

\begin{titlepage}
\begin{center}

\includegraphics[width=0.15\textwidth]{Pyrforos3.png}\\[1cm]
\textsc{\LARGE Εθνικό Μετσόβιο Πολυτεχνείο}\\[1.5cm]

\Large{ 4η Γραπτή Άσκηση }\\[0.5cm]

% Title
\begin{doublespace}
\HRule \\[0.4cm]
{\huge \bfseries
Αλγόριθμοι \& Πολυπλοκότητα
}\\[0.4cm]
\end{doublespace}

\HRule \\[1.5cm]

\begin{minipage}{0.4\textwidth}
\begin{flushleft} \large
\emph{Σπουδαστής:} \\
Διονύσης \textsc{Ζήνδρος} (06601)\\
\textlatin{\textless dionyziz@gmail.com\textgreater}
\end{flushleft}
\end{minipage}
\begin{minipage}{0.4\textwidth}
\begin{flushright} \large
\emph{Διδάσκοντες:} \\
Στάθης \textsc{Ζάχος}\\
Δημήτρης \textsc{Φωτάκης}
\end{flushright}
\end{minipage}

\vfill

{\large 13 Φεβρουαρίου 2012}
\end{center}
\end{titlepage}

\section*{Άσκηση 1}
Ο αλγόριθμος χρωματίζει τις κορυφές του γράφου ως κορυφές που κερδίζουν ή που χάνουν. Μία κορυφή χρωματίζεται ως κορυφή που χάνει αν ο παίκτης που παίζει όταν βρισκόμαστε σε αυτή χάνει ανεξαρτήτως της στρατηγικής του υπό την προϋπόθεση ότι ο αντίπαλος είναι λογικός. Μία κορυφή χρωματίζεται ως κορυφή που κερδίζει αν υπάρχει στραγητική για τον παίκτη που παίζει όταν βρισκόμαστε σε αυτή την κορυφή έτσι ώστε αυτός να μπορέσει να κερδίσει ακόμη και αν ο αντίπαλος είναι λογικός.

Ο χρωματισμός γίνεται ως ακολούθως. Δεδομένης μιας τοπολογικής διάταξης του γράφου, ξεκινάμε από τις κορυφές που δεν έχουν καμία εξερχόμενη ακμή. Αυτές τις χρωματίζουμε ως κορυφές που χάνουν. Στη συνέχεια, σε κάθε βήμα, χρωματίζουμε όλες εκείνες τις κορυφές που οδηγούν σε κορυφές που έχουμε ήδη χρωματίσει. Ως κορυφές που χάνουν χρωματίζουμε όλες τις κορυφές που οδηγούν μόνο σε κορυφές που κερδίζουν. Ως κορυφές που κερδίζουν χρωματίζουμε εκείνες που οδηγούν σε τουλάχιστον μία που χάνει.

Μετά το τέλος της εκτέλεσης του αλγορίθμου, αν η κορυφή $s$ είναι κορυφή που κερδίζει, τότε υπάρχει στρατηγική για τον $A$ που να τον κάνει να κερδίζει ανεξάρτητα από τη στρατηγική του παίκτη $B$.

\section*{Άσκηση 2}
\subsection*{α)} Ο αλγόριθμος ξεκινάει αφαιρώντας όλες τις ακμές που έχουν βάρος μεγαλύτερο από το ντεπόζιτο του αυτοκινήτου. Αυτές δεν μπορούν να χρησιμοποιηθούν σε κάθε περίπτωση. Οι υπόλοιπες μπορούν όλες να χρησιμοποιηθούν, συνεπώς στη συνέχεια τρέχουμε τον αλγόριθμο \textlatin{BFS} και ελέγχουμε αν από το $s$ είναι δυνατόν να φτάσουμε στο $t$.

\subsection*{β)} Αυτή τη φορά ο αλγόριθμος θα πρέπει να υπολογίζει και το μονοπάτι από την $s$ στην $t$ με τη μικρότερη αυτονομία. Δηλαδή αν για κάθε μονοπάτι χαρακτηρίσουμε ως βάρος του το βάρος της μέγιστης ακμής που περιέχει, αναζητούμε εκείνο το μονοπάτι που έχει, ανάμεσα σε όλα τα μονοπάτια, το ελάχιστο βάρος. Αυτό δεν είναι εφικτό με \textlatin{BFS}, καθώς ενδέχεται διαδρομές που αποτελούνται από μεγαλύτερο πλήθος ακμών να έχουν μικρότερες απαιτήσεις σε αυτονομία. Γι' αυτό θα χρησιμοποιηθεί ο αλγόριθμος του \textlatin{Prim} ώστε να υπολογιστεί το μονοπάτι με το ελάχιστο βάρος.

Αφού τρέξουμε τον αλγόριθμο του \textlatin{Prim}, η απάντηση είναι να διασχίζουμε το μονοπάτι από την κορυφή $s$ στην κορυφή $t$ στο \textlatin{MST}. Το κόστος του μονοπατιού τότε μπορεί να αναφερθεί ως η μέγιστη αυτονομία που είναι απαραίτητη για να φτάσουμε από την $s$ στην $t$. 

\section*{Άσκηση 3}
\subsection*{α)}
\begin{lemma}
Σε ένα \textlatin{MST} $T$ ενός συνεκτικού μη κατευθυνόμενου γράφου $G = (V, E, w)$ με $\forall e \in E: w(e) > 0$ ισχύει ότι $\forall (u, v) \in T: d(u, v) = w(u, v)$.
\end{lemma}
\begin{proof}
Έστω $e = (u, v)$. Υποθέτουμε ότι $d(u, v) < w(e)$. Τότε το συντομότερο μονοπάτι από τη $u$ στη $v$ θα αποτελείται από τουλάχιστον δύο ακμές και δεν θα περιέχει την ακμή $e$. Έστω το κόψιμο $S = T \setminus e$ που χωρίζει το γράφο σε δύο ανεξάρτητα σύνολα. Τότε το συντομότερο μονοπάτι από τη $u$ στη $v$ θα πρέπει να διασχίζει το $S$ με κάποια ακμή $f \neq e$ και έστω ότι περιέχει ακόμη κάποια άλλη ακμή $g$. Επειδή το $T$ είναι \textlatin{MST}, θα είναι $w(f) \leq w(e)$. Όμως είναι $w(g) > 0$ και γνωρίζουμε ότι $d(u, v) \geq w(f) + w(g) > w(e)$. Αυτό αποτελεί αντίφαση, και άρα πράγματι η ακμή $e$ είναι το συντομότερο μονοπάτι από το $u$ στο $v$. 
\end{proof}

\subsection*{β)} Το κόστος μίας διαμέρισης σε δύο υποσύνολα θα είναι ίσο με το βάρος της ακμής του \textlatin{MST} που συνδέει τα ανεξάρτητα υποσύνολά της.

Συνεπώς το μέγιστο βάρος ανάμεσα σε όλες τις διαμερίσεις θα προκύπτει από το κόστος της μέγιστης ακμής του \textlatin{MST}. Έτσι ο αλγόριθμος, αφού κατασκευάσει το \textlatin{MST} χρησιμοποιώντας τον αλγόριθμο του \textlatin{Kruskal} σε χρόνο $O( |E| \log{|V|} )$, στη συνέχεια επιλέγει τη μέγιστη ακμή σε χρόνο $O( |E| )$.

\section*{Άσκηση 4}
\subsection*{α)} Το πρόβλημα ανάγεται στο \textlatin{max-flow} με την εξής μοντελοποίηση. Κατασκευάζουμε ένα γράφο με μία πηγή $s$ και μία καταβόθρα $t$. Κάθε κάστρο και κάθε ιππότης μοντελοποιούνται από έναν κόμβο ο καθένας. Τοποθετούμε ακμές βάρους $1$ από κάθε ιππότη προς τα κάστρα που τον προτιμούν. Τοποθετούμε επίσης ακμές από την πηγή προς κάθε ιππότη, κόστους $c_i$ η καθεμία. Τέλος, τοποθετούμε μία ακμή από κάθε κάστρο προς την καταβόθρα κόστους $1$, και από την καταβόθρα προς την πηγή μία ακμή κόστους $\infty$.

Τότε στην έξοδο του $max-flow$, κάθε ακμή που συνδέει κάποιο κάστρο με κάποιον ιππότη που επιλέχθηκε αντιστοιχεί σε μία ανάθεση κάστρου σε ιππότη. Σε περίπτωση που το $max-flow$ είναι μικρότερο του αριθμού των κάστρων σημαίνει ότι κάποιο κάστρο δεν ανατέθηκε σε κανέναν ιππότη και συνεπώς η ανάθεση είναι αδύνατη.

\subsection*{β)}
\subsection*{γ)} Αν μοντελοποιήσουμε την κατάσταση με ένα γράφο όπου κάθε ιππότης είναι ένας κόμβος και κάθε διαμάχη είναι μία ακμή, το πρόβλημα ανάγεται στο πρόβλημα \textlatin{maximum independent-set}. Συγκεκριμένα, η απάντηση του \textlatin{independent-set} καταδεικνύει τους ιππότες που δεν θα εξοριστούν. Επειδή μάλιστα κάθε στιγμιότυπο του \textlatin{independent-set} μπορεί να αναχθεί σε πρόβλημα συγκρούσεων ιπποτών, το πρόβλημα του Βασιλιά είναι \textlatin{NP-complete} και άρα μάλλον δυσεπίλυτο.

\section*{Άσκηση 5}

\end{document}
